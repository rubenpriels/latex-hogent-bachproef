%---------- Inleiding ---------------------------------------------------------

% TODO: Is dit voorstel gebaseerd op een paper van Research Methods die je
% vorig jaar hebt ingediend? Heb je daarbij eventueel samengewerkt met een
% andere student?
% Zo ja, haal dan de tekst hieronder uit commentaar en pas aan.

%\paragraph{Opmerking}

% Dit voorstel is gebaseerd op het onderzoeksvoorstel dat werd geschreven in het
% kader van het vak Research Methods dat ik (vorig/dit) academiejaar heb
% uitgewerkt (met medesturent VOORNAAM NAAM als mede-auteur).
% 

\section{Inleiding}%
\label{sec:inleiding}

Bedrijven zoals winkels en restaurants zijn sterk afhankelijk van een goede locatiekeuze. De populariteit van een locatie speelt namelijk een grote rol in het succes van een onderneming. Als een bedrijf zich vestigt op een locatie met lage bezoekersaantallen, dan leidt dit ook tot een lager klantenaantal en bijgevolg ook een lagere omzet. Talloze bedrijven hebben echter onvoldoende inzicht in de huidige en toekomstige populariteit van een locatie. Hierdoor nemen deze bedrijven verkeerde beslissingen bij nieuwe vestigingen te openen of reclamecampagnes te starten.

Dit onderzoek richt zich daarom op de hoofdvraag: “Hoe kan artificiële intelligentie worden ingezet om de populariteit van een point-of-interest te voorspellen op basis van census data?" Om deze hoofdvraag te beantwoorden, wordt het onderzoek opgedeeld in drie deelvragen. Welke technieken zijn er beschikbaar voor dataverzameling? Welke deep learning-modellen zijn geschikt om POI-populariteit te voorspellen? Welke technieken bestaan er om de prestaties van zulke deep learning-modellen te verbeteren? De gewenste oplossing moet bedrijven in staat stellen om op basis van de voorspellingen een strategische locatie te kiezen voor een vestiging te openen of een marketingcampagne te starten. De oplossing moet dus technisch haalbaar zijn en een foutmarge van maximaal 10\% hebben voor voorspellingen binnen 1 jaar.

\bigskip
%---------- Stand van zaken ---------------------------------------------------

\section{Literatuurstudie}%
\label{sec:literatuurstudie}

Hier beschrijf je de \emph{state-of-the-art} rondom je gekozen onderzoeksdomein, d.w.z.\ een inleidende, doorlopende tekst over het onderzoeksdomein van je bachelorproef. Je steunt daarbij heel sterk op de professionele \emph{vakliteratuur}, en niet zozeer op populariserende teksten voor een breed publiek. Wat is de huidige stand van zaken in dit domein, en wat zijn nog eventuele open vragen (die misschien de aanleiding waren tot je onderzoeksvraag!)?

Je mag de titel van deze sectie ook aanpassen (literatuurstudie, stand van zaken, enz.). Zijn er al gelijkaardige onderzoeken gevoerd? Wat concluderen ze? Wat is het verschil met jouw onderzoek?

Verwijs bij elke introductie van een term of bewering over het domein naar de vakliteratuur, bijvoorbeeld~\autocite{Hykes2013}! Denk zeker goed na welke werken je refereert en waarom.

Draag zorg voor correcte literatuurverwijzingen! Een bronvermelding hoort thuis \emph{binnen} de zin waar je je op die bron baseert, dus niet er buiten! Maak meteen een verwijzing als je gebruik maakt van een bron. Doe dit dus \emph{niet} aan het einde van een lange paragraaf. Baseer nooit teveel aansluitende tekst op eenzelfde bron.

Als je informatie over bronnen verzamelt in JabRef, zorg er dan voor dat alle nodige info aanwezig is om de bron terug te vinden (zoals uitvoerig besproken in de lessen Research Methods).

% Voor literatuurverwijzingen zijn er twee belangrijke commando's:
% \autocite{KEY} => (Auteur, jaartal) Gebruik dit als de naam van de auteur
%   geen onderdeel is van de zin.
% \textcite{KEY} => Auteur (jaartal)  Gebruik dit als de auteursnaam wel een
%   functie heeft in de zin (bv. ``Uit onderzoek door Doll & Hill (1954) bleek
%   ...'')

Je mag deze sectie nog verder onderverdelen in subsecties als dit de structuur van de tekst kan verduidelijken.

\clearpage
%---------- Methodologie ------------------------------------------------------
\section{Methodologie}%
\label{sec:methodologie}

\paragraph{Fase 1: Literatuurstudie.}
Het doel van de eerste fase is om een inzicht te krijgen in bestaande deep learning-modellen en technieken voor dataverzameling en model optimalisatie. Daarnaast worden de specifieke voorwaarden opgesteld van het deep learning-model dat voldaan moet zijn om de oplossing als een succes te beschouwen. Hiervoor wordt een literatuuronderzoek uitgevoerd naar verschillende onderzoekspapers om geschikte deep learning-modellen te identificeren zoals MLP die tot een mogelijke oplossing kunnen leiden. Deze papers behandelen zowel real-world use cases waar deze modellen gebruikt worden als reviews over de architectuur van specifieke modellen zoals transformers. Aan het einde van deze fase is een uitgebreide documentatie van mogelijke deep learning-modellen en technieken voor dataverzameling en model optimalisatie opgesteld. Daarnaast is de maximale voorspellingsfout en modelnauwkeurigheid duidelijk afgebakend. De maximale voorspellingsfout geeft aan hoeveel de voorspellingen maximaal mogen afwijken van de werkelijke waarden en de modelnauwkeurigheid geeft aan hoe goed het model de variatie in de data verklaart. De literatuurstudie vindt plaats twee dagen per week over een periode van zeven weken. Er wordt verwacht dat deze deliverables afgewerkt zijn op 23 maart.

\paragraph{Fase 2: Data verzamelen.}
Het doel van de tweede fase is om voldoende diverse data te verzamelen om de deep learning-modellen mee te trainen in de volgende fase. Hierbij worden zowel openbare datasets als door bedrijf aangeleverde datasets gebruikt. Daarnaast worden technieken uit de eerste fase toegepast om data schaarste tegen te gaan zoals synthetische data en x. Aan het einde van deze fase is er een dataset beschikbaar van censusdata waarmee de deep learning-modellen uit de volgende fase mee getraind kunnen worden. Deze fase vindt plaats twee dagen per week over een periode van vijf weken. Er wordt verwacht dat deze deliverables afgewerkt zijn op 6 april.

\paragraph{Fase 3: deep learning-model implementatie.}
Het doel van de derde fase is om deep learning-modellen te implementeren op basis van de bevindingen uit de literatuurstudie. Deze modellen kunnen op basis van censusdata de populariteit van een POI voorspellen. Eerst worden de standaard deep learning-modellen geprogrammeerd die in fase 1 zijn onderzocht. Vervolgens worden deze modellen geoptimaliseerd door de in de literatuurstudie beschreven technieken voor model optimalisatie toe te passen. Deze deep learning-modellen worden getraind op de dataset uit de vorige fase. Aan het einde van deze fase zijn werkende deep learning-modellen ontwikkeld op basis van de literatuurstudie. Deze kunnen de populariteit van een POI voorspellen. De implementatie van de deep learning-modellen vindt plaats twee dagen per week over een periode van vijf weken. Er wordt verwacht dat deze deliverables afgewerkt zijn op 13 april.

\paragraph{Fase 4: Experimenten uitvoeren.}
Het doel van de vierde fase is om de deep learning-modellen te testen door middel van experimenten. Alle deep learning-modellen die zijn ontwikkeld tijdens de Proof of Concept zullen dezelfde experimenten ondergaan. Ze worden geëvalueerd aan de hand van verschillende prestatiemaatstaven waaronder RMSE, MAE en R². De experimenten omvatten voorspellingen op diverse scenario's zoals reguliere gevallen, situaties met beperkte data, tijdsperiodes met evenementen op een specifieke locatie en feestdagen. Aan het einde van deze fase is elk deep learning-model geëvalueerd, wat resulteert in een gedetailleerd rapport van hun prestaties. Deze experimenten vinden plaats twee dagen per week over een periode van vier weken. Er wordt verwacht dat deze deliverables afgewerkt zijn op 27 april.

\paragraph{Fase 5: Evaluatie van deep learning-modellen.}
In de laatste fase worden de prestaties van de verschillende deep learning-modellen geëvalueerd voor gebruik in real world cases waarbij bedrijven een locatie kunnen kiezen aan de hand van deze voorspellingen. De resultaten uit de vorige fase worden zowel individueel geanalyseerd als vergeleken met de resultaten van de andere deep learning-modellen. Hierdoor wordt nagegaan welke modellen beter zijn in specifieke situaties, maar ook welk model in het algemeen het beste presteert op de vooraf gedefinieerde criteria. De deliverables van deze fase is een uitgebreide rapportage waarin alle deep learning-modellen worden geanalyseerd op basis van de vooraf gedefinieerde criteria en hoe ze presteren ten opzichte van andere modellen. Hieruit wordt geconcludeerd welk deep learning-model het meest geschikt is om de populariteit van POI’s te voorspellen en of deze daadwerkelijk voldoet aan de vereisten die in fase 1 werden vastgelegd. Tot slot worden mogelijke beperkingen gedocumenteerd zoals slechtere prestaties bij uitzonderlijke situaties samen met aanbevelingen voor verbeteringen. De evaluatie van de deep learning-modellen vindt plaats twee dagen per week over een periode van drie weken. Er wordt verwacht dat deze deliverables afgewerkt zijn op 18 mei.

\begin{figure*}
  \centering
  \includegraphics[width=\textwidth]{img/Gantt_diagram.png}
  \caption{\label{fig:gantt}Gantt diagram met de verschillende fasen en milestones van de bachelorproef.}
\end{figure*}

% \begin{figure*}
%     \centering
%     \includegraphics[width=\textwidth]{img/Gantt_diagram.png}
%     \caption{\label{fig:gantt}Gantt diagram met de verschillende fasen en milestones van de bachelorproef.}
% \end{figure*}

%---------- Verwachte resultaten ----------------------------------------------
\section{Verwacht resultaat, conclusie}%
\label{sec:verwachte_resultaten}

Hier beschrijf je welke resultaten je verwacht. Als je metingen en simulaties uitvoert, kan je hier al mock-ups maken van de grafieken samen met de verwachte conclusies. Benoem zeker al je assen en de onderdelen van de grafiek die je gaat gebruiken. Dit zorgt ervoor dat je concreet weet welk soort data je moet verzamelen en hoe je die moet meten.

Wat heeft de doelgroep van je onderzoek aan het resultaat? Op welke manier zorgt jouw bachelorproef voor een meerwaarde?

Hier beschrijf je wat je verwacht uit je onderzoek, met de motivatie waarom. Het is \textbf{niet} erg indien uit je onderzoek andere resultaten en conclusies vloeien dan dat je hier beschrijft: het is dan juist interessant om te onderzoeken waarom jouw hypothesen niet overeenkomen met de resultaten.

